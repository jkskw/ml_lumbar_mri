\chapter{Praca z szablonem}
\section{Środowisko}
Szablon powinien dać się skompilować w dowolnym środowisku, w którym zainstalowano system \LaTeX. 
Zalecaną konfiguracją, która działa w systemie Windows, jest: \texttt{MiKTeX} (windowsowa dystrybucja latexa) + \texttt{TeXnicCenter} (środowisko do edycji i kompilacji projektów latexowych) + \texttt{SumatraPDF} (przeglądarka pdfów z nawigacją zwrotną) + \texttt{JabRef} (opcjonalny edytor bazy danych bibliograficznych). Narzędzia te można pobrać ze stron internetowych, których adresy zamieszczono w tabeli~\ref{tab:narzedzia}. 
\begin{table}[htb] \small
\centering
\caption{Wykaz zalecanych narzędzi do pracy z wykorzystaniem szablonu (na dzień 09.02.2021)}
\label{tab:narzedzia}
\begin{tabularx}{\linewidth}{|c|c|X|p{5.5cm}|} \hline\
Narzędzie & Wersja & Opis & Adres \\ \hline\hline
\texttt{MiKTeX} & 22.12 & Zalecana jest instalacja \texttt{Basic MiKTeX} 32 lub 64 bitowa. Brakujące pakiety będą się doinstalowywać podczas kompilacji projektu &
\url{http://miktex.org/download} \\ \hline
\texttt{TeXnicCenter} & 2.02 &  Można pobrać 32 lub 64 bitową wersję & \url{http://www.texniccenter.org/download/} \\ \hline
\texttt{SumatraPDF} & 3.4.6 & Można pobrać 32 lub 64 bitową wersję & \url{http://www.sumatrapdfreader.org/download-free-pdf-viewer.html} \\ \hline
\texttt{JabRef} & 5.7	 & Rozwijane w JDK 15, ma własny instalator i wersję przenośną & \url{http://www.fosshub.com/JabRef.html} \\ \hline
\end{tabularx}
\end{table}

Wspomniana nawigacja między \texttt{TeXnicCenter} a \texttt{SumatraPDF} polega na przełączaniu się pomiędzy tymi narzędziami z zachowaniem kontekstu położenia kursora. Czyli edytując tekst w \texttt{TeXnicCenter} po kliku na ikonce podglądu można przeskoczyć do odpowiedniego miejsca w pdfie wyświetlanym przez \texttt{SumatraPDF}. Podwójne kliknięcie zaś w pdfie widocznym w \texttt{SumatraPDF} ustawi kursor we właściwym akapicie w edytorze tekstu \texttt{TeXnicCenter}. O konfiguracji obu narzędzi do takiej współpracy napisano na stronie \url{http://tex.stackexchange.com/questions/116981/how-to-configure-texniccenter-2-0-with-sumatra-2013-2014-2015-version} (w sieci można znaleźć również inne materiały na ten temat).

Szablon można również kompilować za pomocą innych narzędzi i środowisk (np.\ za pomocą multiplatformowego \texttt{TexStudio} oraz różnych dystrybucji systemu \LaTeX). Może to jednak czasem wymagać zmiany sposobu kodowania plików i korekty deklaracji kodowania znaków w dokumencie głównym (co opisano dalej). 
Można go też zaadoptować do wymogów \LaTeX-owych edytorów i kompilatorów działających w trybie on-line, oferowanych na przykład w usłudze \texttt{Overleaf} (\url{https://www.overleaf.com/}). 

Choć usługa \texttt{Overleaf} ma niewątpliwie wiele zalet (pozwala na edycję kodu przez wielu użytkowników jednocześnie), to podczas pracy nad długimi dokumentami przegrywa ona ze wspomnianymi zintegrowanymi środowiskami \texttt{TeXnicCenter} czy \texttt{TexStudio}. W \texttt{Overleaf} nie można wyświetlić struktury całego projektu (a jedynie listę plików projektowych oraz strukturę bieżąco edytowanego dokumentu), nie można uruchomić wyszukiwania we wszystkich plikach projektowych (nie ma opcji  \texttt{Find in Files ...}, która bardzo się przydaje, gdy nie wiadomo, w~którym pliku jest wyszukiwany tekst). 

\texttt{Overleaf} wymaga użycia kodowania UTF8. Czyli aby dało się wczytać niniejszy szablon pracy dyplomowej do tego środowiska, pliki szablonu muszą być przekodowane, muszą też być zmienione opcje pakietu \texttt{inputenc} oraz odpowiednio zadeklarowane \texttt{literate} (aby dało się używać polskich znaków na listingach). Aby sprawę uprościć przygotowano dwie wersje szablonu: w kodowaniu ANSI oraz UTF8.

Choć w \texttt{Overleaf} istnieje możliwość skorzystania z wersjonowania (system ten integruje się z repozytorium \texttt{git}), to wersjonowanie to nie działa perfekcyjnie. Dlatego \textbf{zaleca się korzystanie ze środowisk zintegrowanych zainstalowanych lokalnie, czemu towarzyszyć ma wersjonowanie z pomocą wybranego zdalnego repozytorium (\texttt{gitlab}, \texttt{github}, \texttt{bitbucket})}.

\section{Struktura projektu}
Pisząc pracę w systemie \LaTeX zwykle przyjmuje się jakąś konwencję co do nazewnictwa tworzonych plików, ich położenia oraz powiązań. Przygotowując niniejszy szablon założono, że projekt będzie się składał z pliku głównego, plików z kodem kolejnych rozdziałów i dodatków (włączanych do kompilacji w dokumencie głównym), katalogów z plikami grafik (o nazwach wskazujących na rozdziały, w których grafiki te zostaną wstawione), pliku ze skrótami (opcjonalny), pliku z danymi bibliograficznymi (plik \texttt{dokumentacja.bib}). Taki ,,układ'' zapewnia porządek oraz pozwala na selektywną kompilację rozdziałów. 

Przyjętą konwencję da się opisać jak następuje:
\begin{itemize}
\item Plikiem głównym jest plik \texttt{Dyplom.tex}. To w nim znajdują się deklaracje wszystkich używanych styli, definicje makr oraz ustawień, jak również polecenie \verb+\begin{document}+. W~nim też należy ustawić metadane, które pojawią się na stronie tytułowej (oraz w stopkach).
\item Teksty redagowane są w osobnych plikach. Pliki te zamieszczone są w katalogu głównym (tym samym, co plik \texttt{Dyplom.tex}).
\item W pliku \texttt{streszczenie.tex} powinien pojawić się tekst streszczenia ze słowami kluczowymi (tekst ten oraz słowa kluczowe będzie można wykorzystać do wypełnienia formularzy pojawiających się podczas wysyłania pracy do analizy antyplagiatowej w systemie ASAP)
\item Plik \texttt{skroty.tex} powinien zawierać wykaz użytych skrótów. Można z tego pliku zrezygnować, jeśli liczba stosowanych skrótów jest nieznaczna. 
\item Tekst kolejnych rozdziałów powinien pojawić się w plikach o nazwach zawierających numery tych rozdziałów. Według przyjętej konwencji \texttt{rozdzial01.tex} to plik pierwszego rozdziału, \texttt{rozdzial02.tex} to plik z treścią drugiego rozdziału itd. 
\item Teksty dodatków mają być zapisywane w osobnych plikach o nazwach zawierających literę dodatku. Pliki te, podobnie do plików z tekstem rozdziałów, zamieszczane są w katalogu głównym. I~tak \texttt{dodatekA.tex} oraz \texttt{dodatekB.tex} to, odpowiednio, pliki z treścią dodatku A oraz dodatku B.
\item Każdemu rozdziałowi i dodatkowi towarzyszy katalog przeznaczony do składowania dołączanych grafik. I tak \texttt{rys01} to katalog na pliki z grafikami dołączanymi do rozdziału pierwszego, \texttt{rys02} to katalog na pliki z grafikami dołączanymi do rozdziału drugiego itd.
Podobnie \texttt{rysA} to katalog na pliki z grafikami dołączanymi w dodatku A itd.
\item W katalogu głównym zamieszczany jest plik \texttt{dokumentacja.bib} zawierający bazę danych bibliograficznych.
\item Jeśli praca nad dokumentem odbywa się w \texttt{TeXnicCenter}, to zgodnie z wymogami tego narzędzia powinien istnieć jeszcze dodatkowy plik projektu. Plik ten pełni podobną rolę jak plik solucji wykorzystywany w zintegrowanych środowiskach programowania. Co prawda plik projektu nie jest wymagany do \LaTeX-owej kompilacji (wystarczy kompilować plik główny), niemniej pozwala zapamiętać ustawienia środowiska (w tym ustawienia językowe potrzebne do sprawdzania poprawności wyrazów -- patrz następny podrozdział).  Plik projektu dostarczony w szablonie ma nazwę \texttt{Dyplom.tcp}. 
\end{itemize}

\begin{table}[htb]
\centering\small
\caption{Pliki źródłowe szablonu oraz wyniki kompilacji}
\label{tab:szablon}
\begin{tabularx}{\linewidth}{|p{.55\linewidth}|X|}\hline
Źródła & Wyniki kompilacji \\ \hline\hline
\verb?Dokument.tex? - dokument główny\newline
\verb?Dokument.tcp? -- plik projektu \verb+TeXnicCenter+\newline
\verb?streszczenie.tex? -- plik streszczenia\newline
\verb?skroty.tex? -- plik ze skrótami\newline
\verb?rozdzial01.tex? -- plik rozdziału \texttt{01}\newline
\verb?...?\newline
\verb?dodatekA.tex? -- plik dodatku \texttt{A}\newline
\verb?...?\newline
\verb?rys01? -- katalog na rysunki do rozdziału \texttt{01}\newline
\verb?   |- fig01.png? -- plik grafiki\newline
\verb?   |- ...?\newline
\verb?...?\newline
\verb?rysA? -- katalog na rysunki do dodatku \texttt{A}\newline
\verb?   |- fig01.png? -- plik grafiki\newline
\verb?   |- ...?\newline
\verb?...?\newline
\verb?dokumentacja.bib? -- plik danych bibliograficznych\newline
\verb?Dyplom.ist? -- plik ze stylem indeksu\newline
\verb?by-nc-sa.png? -- plik z ikonami CC\newline
 &
\verb?Dyplom.bbl?\newline
\verb?Dyplom.blg?\newline
\verb?Dyplom.ind?\newline
\verb?Dyplom.idx?\newline
\verb?Dyplom.lof?\newline
\verb?Dyplom.log?\newline
\verb?Dyplom.lot?\newline
\verb?Dyplom.out?\newline
\verb?Dyplom.pdf? -- dokument wynikowy\newline
\verb?Dyplom.syntex?\newline
\verb?Dyplom.toc?\newline
\verb?Dyplom.tps?\newline
\verb?*.aux?\newline 
\verb?Dyplom.synctex?\newline\\
\hline
\end{tabularx}
\end{table}

\section{Kodowanie znaków}
System \LaTeX obsługuje wielojęzyczność. Można tworzyć w nim dokumenty z tekstem zawierającym różne znaki diakrytyczne. 
Należy jednak zdawać sobie sprawę, w jaki sposób znaki te są obsługiwane. Na kodowanie znaków należy patrzeć z dwóch perspektyw: perspektywy edytowania kodu źródłowego oraz perspektywy kodowania dokumentu wynikowego i użytych czcionek.

Kod latexowy może być edytowany w dowolnym edytorze tekstów. Zastosowane kodowanie znaków w tym edytorze musi być znane \LaTeX-owi, inaczej kompilacja tego kodu się nie powiedzie. Informację o tym kodowaniu przekazuje się w opcjach pakietu \texttt{inputenc}. Niniejszy szablon przygotowano w systemie Windows, a \LaTeX-owe źródła umieszczono w plikach ANSI z użyciem strony kodowej cp1250. Dlatego w poleceniu \verb+\usepackage[cp1250]{inputenc}+ jako opcję wpisano \texttt{cp1250}.


Szablon można wykorzystać również przy innych kodowaniach i w innych systemach. Jednak wtedy konieczna będzie korekta dokumentu \texttt{Dyplom.tex} odpowiednio do wybranego przypadku. Korekta ta polegać ma na zamianie polecenia \verb+\usepackage[cp1250]{inputenc}+  na polecenie \verb+\usepackage[utf8]{inputenc}+ oraz konwersji znaków i zmiany kodowania istniejących plików ze źródłem latexowego kodu (plików o rozszerzeniu \texttt{*.tex} oraz \texttt{*.bib}).

Kodowanie znaków jest istotne również przy edytowaniu bazy danych bibliograficznych (pliku \texttt{dokumentacja.bib}). Aby \texttt{bibtex} poprawnie interpretował polskie znaki plik \texttt{dokumentacja.bib} powinien być zakodowany w ANSI, CR+LF (dla ustawień jak w szablonie). 
W szczególności, jeśli ominąć chce się problem kodowania, polskie znaki w bazie danych bibliograficznych można zastąpić odpowiednią notacją: \verb|\k{a}| \verb|\'c| \verb|\k{e}| \verb|\l{}| \verb|\'n| \verb|\'o| \verb|\'s| \verb|\'z| \verb|\.z| \verb|\k{A}| \verb|\'C| \verb|\k{E}| \verb|\L{}| \verb|\'N| \verb|\'O| \verb|\'S| \verb|\'Z| \verb|\.Z|. 

Samo kodowanie plików może być źródłem paru problemów. Chodzi o to, że użytkownicy pracujący z edytorami tekstów pod linuxem mogą generować pliki zakodowane w UTF8 bez BOM (lub z BOM -- co nie jest zalecane), a pod windowsem -- pliki ANSI ze znakami ze strony kodewej \texttt{cp1250}. Z takimi plikami różne edytory różnie sobie radzą. W szczególności edytor \texttt{TeXnicCenter} podczas otwierania plików może potraktować jego zawartość jako UTF8 lub ANSI -- prawdopodobnie interpretuje z jakim kodowaniem ma do czynienia na podstawie obecności w pliku znaków specjalnych. Bywa, że choć wszystko w \texttt{TeXnicCenter} wygląda na poprawne, to jednak kompilacja \LaTeX-owa ,,nie idzie''. Problemem mogą być właśnie pierwsze bajty, których nie widać w edytorze. 

Do konwersji kodowania można użyć edytor \texttt{Notepad++} (jest tam opcja ,,konwertuj'' -- nie mylić z~opcją ,,koduj'', która przekodowuje znaki, jednak nie zmienia sposobu kodowania pliku).

Jeśli chodzi o drugą perspektywę, tj.\ kodowanie znaków w dokumencie wynikowym, to sprawa jest bardziej skomplikowana. Wiąże się z nią zarządzanie czcionkami, definiowanie mapowania itp. 
\textbf{Szablon przygotowano tak, by wynikowy dokument zawierał polskie znaki diakrytyczne, które nie są zlepkami literki i ogonka.}

\section{Kompilacja szablonu}
Kompilację szablonu można uruchamiać na kilka różnych sposobów. Wszystko zależy od używanego systemu operacyjnego, zainstalowanej na nim dystrybucji \LaTeX-a oraz dostępnych narzędzi. Zazwyczaj kompilację rozpoczyna się wydając polecenie z linii komend lub uruchamia się ją za pomocą narzędzi zintegrowanych środowisk.

Kompilacja z linii komend polega na uruchomieniu w katalogu, w którym rozpakowano źródła szablonu, następującego polecenia:
\begin{lstlisting}[basicstyle=\ttfamily]
> pdflatex Dyplom.tex
\end{lstlisting}
gdzie \texttt{pdflatex} to nazwa kompilatora, zaś \texttt{Dyplom.tex} to nazwa głównego pliku redagowanej pracy. 
W przypadku korzystania ze środowiska \texttt{TeXnicCenter} należy otworzyć dostarczony w~szablonie plik projektu \texttt{Dyplom.tcp}, a następnie uruchomić kompilację narzędziami dostępnymi w pasku narzędziowym.

Aby poprawnie wygenerowały się wszystkie referencje (spis treści, odwołania do tabel, rysunków, pozycji literaturowych, równań itd.) kompilację \texttt{pdflatex} należy wykonać dwukrotnie, a~czasem nawet trzykrotnie. Wynika to z konieczności zapamiętywania pośrednich wyników kompilacji i ich wykorzystywania w kolejnych przebiegach. Tak dzieje się przy generowaniu odwołań do pozycji literaturowych oraz tworzeniu ich wykazu). 

Wygenerowanie danych bibliograficznych zapewnia kompilacja \texttt{bibtex} uruchamiana po kompilacji \texttt{pdfltex}. Można to zrobić z linii komend:
\begin{lstlisting}[basicstyle=\ttfamily]
> bibtex Dyplom
\end{lstlisting}
lub wybierając odpowiednią pozycję z paska narzędziowego wykorzystywanego środowiska. Po kompilacji za pomocą \texttt{bibtex} na dysku pojawi się plik \texttt{Dyplom.bbl}. Dopiero po kolejnych dwóch kompilacjach \texttt{pdflatex} dane z tego pliku zostaną odpowiednio przetworzone i zrenderowane w wygenerowanym dokumencie. Tak więc po każdym wstawieniu nowego cytowania w kodzie dokumentu uzyskanie poprawnego formatowania dokumentu wynikowego wymaga powtórzenia następującej sekwencji kroków kompilacji:
\begin{lstlisting}[basicstyle=\ttfamily]
> pdflatex Document.tex
> bibtex Document
> latex Document.tex
> latex Document.tex
\end{lstlisting}
Szczegóły dotyczące przygotowania danych bibliograficznych oraz zastosowania cytowań przedstawiono w podrozdziale \ref{sec:literatura}.

W głównym pliku zamieszczono polecenia pozwalające sterować procesem kompilacji poprzez włączanie bądź wyłączanie kodu źródłowego poszczególnych rozdziałów. Włączanie kodu do kompilacji zapewniają instrukcje \verb+\include+ oraz \verb+\includeonly+. Pierwsza z nich pozwala włączyć do kompilacji kod wskazanego pliku (np.\ kodu źródłowego pierwszego rozdziału \verb+\chapter{Wstęp}
\section{Wprowadzenie}
Niniejszy dokument powstał z myślą o ujednoliceniu sposobu redagowania prac dyplomowych magisterskich i inżynierskich. Cel ten starano się osiągnąć poprzez sformułowanie reguł redakcyjnych oraz dostarczenie gotowego do wykorzystania kodu źródłowego, przygotowanego do kompilacji w systemie \LaTeX{}. W dokumencie zebrano więc zalecenia i uwagi o charakterze technicznym (dotyczące takich zagadnień, jak na przykład: formatowanie tekstu, załączanie rysunków, układ strony) oraz redakcyjnym (odnoszące się do stylu wypowiedzi, sposobów tworzenia referencji itp.). Ponadto skomentowano stworzony kod źródłowy, by na podstawie przekazanych wskazówek lepiej można było zrozumieć znaczenie użytych komend. W efekcie prezentowany dokument (a raczej zestaw dokumentów wchodzących w skład \LaTeX{}-owego projektu) pełnić może rolę szablonu, w którym wystarczy zmienić treść, aby po kompilacji uzyskać dobrze sformatowaną pracę dyplomową w~postaci dokumentu \texttt{pdf}.  



Szablon przygotowano do kompilacji narzędziem \texttt{pdflatex} należącym do dystrybucji systemu \LaTeX. Aby skorzystać z szablonu należy wcześniej zainstalować ten system  bądź też skorzystać z usług kompilacji \LaTeX-owych źródeł dostępnych on-line (jak \texttt{OverLeaf}). Na pierwszy rzut oka kod źródłowy szablonu (w szczególności głównego dokumentu \texttt{Dyplom.tex}) może wydać się nieco skomplikowany. Zdecydowano bowiem, by zamiast tworzyć osobną klasę dokumentu lepiej będzie wykorzystać jakąś istniejącą klasę, oferującą zestaw komend ułatwiających składanie tekstu. Wybór padł na klasę \texttt{memoir}. W efekcie szablon utworzono jako sparametryzowaną instancję tej klasy. 
Samo zaś użycie szablonu jest dość proste. Wystarczy podmienić wartości atrybutów komend użytych do zdefiniowania zawartości strony tytułowej (metadane dokumentu: tytuł, autor, promotor, kierunek, specjalność, słowa kluczowe), a później zadbać o~właściwą strukturę reszty dokumentu. 

W szablonie zamieszczono komendy zapewniające dołączenie  do wynikowego dokumentu \texttt{pdf} metadanych z podstawowymi informacjami (tytuł, autor, temat, słowa kluczowe, data). Metadane te będą widoczne we właściwościach dokumentu, gdy zacznie się go przeglądać w jakiejś przeglądarce \texttt{pdf} (np.~\texttt{SumatraPD}F lub \texttt{Acrobat Reader}). Niestety, system \LaTeX{} nie wspiera tworzenia dostępnych plików \texttt{pdf} (zgodnych ze standardami PDF/UA, WCAG 2.0/2.1/2.2).

Szablon pracy dyplomowej nie wiąże się bezpośrednio z tzw.\ ,,Kartą tematu pracy dyplomowej''. Karty tematów są syntetycznymi opisami, które podlegają oficjalnej procedurze zgłaszania, zatwierdzania i wybierania (finalizowanej dokonaniem wpisu studenta na kurs ,,Praca dyplomowa''). Formalnie zawartość kart tematów prac dyplomowych regulowana jest zarządzeniami odpowiedniego Dziekana. Przystępując do redakcji pracy dyplomowej z wykorzystaniem niniejszego szablonu należy pamiętać o obowiązku zachowania zgodności prezentowanych treści z zawartością odpowiedniej karty tematu (merytorycznie musi się wszystko zgadzać).

Zasadniczo tematy prac inżynierskich wiążą się z wykonaniem jakiegoś konkretnego dzieła (produktu). Formułując cel używa się zwrotów:
budowa, implementacja, projekt, przeprowadzenia itp. Rolą dyplomanta (na kierunku informatyka) jest dostarczenie dzieła (przynajmniej w~formie prototypu, spełniającego podstawowe wymagania funkcjonalne). Niniejszy szablon pozwolić ma na opisanie tego dzieła.

Nieco inaczej jest w przypadku tematów prac magisterskich. Tutaj temat wraz z opisem celu i zakresu wyznaczać mają jakiś kierunek badań czy analiz. Od razu nie wiadomo przecież, do czego się dojdzie. A jeśli byłoby wiadomo, to nie byłoby sensu robić badań.  Na tym właśnie polega "piękno" pracy badawczej. 
Zwykle więc podczas formułowania tematów i opisów tego typu prac stosuje się słowa: badanie, analiza, przegląd, charakterystyka itp. Rolą magistranta jest eksploracja wyznaczonego kierunku, dobre uwarunkowanie analizowanych problemów, przeprowadzenie badań, dostarczenie odpowiedzi. Niniejszy szablon w tym przypadku posłużyć ma za ramy, dzięki którym praca może przyjąć formę dokumentu naukowego.

Po kompilacji wynikowy dokument \texttt{Dyplom.pdf} należy załadować do systemu APD USOS (\url{https://apd.usos.pwr.edu.pl/}) celem weryfikacji antyplagiatowej i dalszego procesowania. System ten zmienia nazwę załadowanemu plikowi na taką, w której ukryty jest kod wydziału, kod kierunku, typ pracy i jeszcze parę innych danych. Nazwa ta przyjmuje, przykładowo, następującą postać: \texttt{W4N\_\#\#\#\#\#\#\_W04-ITE-INZ\_W04-ITEP-000P-OSIW7.pdf}, gdzie \texttt{\#\#\#\#\#\#} to miejsce na numer indeksu dyplomanta. Z tej racji trudno przewidzieć, jak ostatecznie należy się odwoływać do pliku zarejestrowanego w systemie. Jest to o tyle istotne, iż podczas składania prac dyplomowych wciąż stosowaną praktyką jest dołączanie płyty CD/DVD z jej wynikami (zawierającej dokumentem \texttt{pdf} z tekstem pracy, jak również kodami źródłowymi stworzonego dzieła, instalatorami itp.). Proszę zajrzeć do dodatku \ref{chap:opis-plyty} po dodatkowe wyjaśnienia.

\section{Układ dokumentu}
W rozdziale pierwszym przedstawiono w zarysie czym jest i czego dotyczy niniejszy dokument (jest to szablon, który można zastosować podczas redagowania pracy dyplomowej inżynierskiej bądź magisterskiej). W rozdziale drugim opisano sposób pracy z szablonem. W kolejnym, trzecim rozdziale, przedstawiono zalecenia dotyczące formatowania dokumentu. Rozdział ten pełni rolę czysto informacyjną (dostarczony szablon zapewnia uzyskanie opisanego tam formatowania).
W rozdziale czwartym zwrócono uwagę na redakcję pracy dyplomowej (od strony edytorskiej i merytorycznej).
Rozdział piąty poświęcono na uwagi techniczne. Ostatni, szósty rozdział, przeznaczono na kilka słów podsumowania oraz ,,lorem ipsum'' -- wygenerowany tekst, pełniący rolę ,,wypełniacza'', wykorzystany w celach poglądowych (jak dzielić dokument na sekcje).
Pracy towarzyszy przykładowy wykaz literatury oraz przykładowe dwa dodatki. 

+). Druga, jeśli zostanie zastosowana, pozwala określić, które z~plików zostaną skompilowane w całości (na przykład kod źródłowy pierwszego i drugiego rozdziału \verb+\includeonly{rozdzial01.tex,rozdzial02.tex}+). Brak nazwy pliku na liście w poleceniu \verb+\includeonly+ przy jednoczesnym wystąpieniu jego nazwy w poleceniu \verb+\include+ oznacza, że w kompilacji zostaną uwzględnione referencje wygenerowane dla tego pliku wcześniej, sam zaś kod źródłowy pliku nie będzie kompilowany. 

W szablonie wykorzystano klasę dokumentu \texttt{memoir} oraz wybrane pakiety. Podczas kompilacji szablonu w \texttt{MikTeXu} wszelkie potrzebne pakiety zostaną zainstalowane automatycznie (jeśli \texttt{MikTeX} zainstalowano z opcją dynamicznej instalacji brakujących pakietów). W przypadku innych dystrybucji \LaTeX-a może okazać się, że pakiety te trzeba doinstalować ręcznie (np.\ pod linuxem z \texttt{TeXLive} trzeba doinstalować dodatkową zbiorczą paczkę, a jeśli ma się menadżera pakietów \LaTeX-owych, to pakiety te można instalować indywidualnie).

Jeśli w szablonie będzie wykorzystany indeks rzeczowy, kompilację źródeł trzeba będzie rozszerzyć o kroki potrzebne na wygenerowanie plików pośrednich \texttt{Dokument.idx} oraz \texttt{Dokument.ind} oraz dołączenia ich do finalnego dokumentu (podobnie jak to ma miejsce przy generowaniu wykazu literatury).
Szczegóły dotyczące generowania indeksu rzeczowego opisano w podrozdziale~\ref{sec:indeks}.

\section{Sprawdzanie poprawności tekstu}
Większość środowisk ułatwiających pisanie \LaTeX-owych dokumentów wspiera sprawdzenie poprawności tekstu (ang.~\emph{spell checking}). Wystarczy odpowiednio je skonfigurować. Niestety, proponowana przez narzędzia korekta nie jest genialna. Bazuje ona na prostym porównywaniu wyrazów (z końcówkami). Nie wbudowano w nią żadnej większej inteligencji. Tak więc proszę nie porównywać jej z korektą oferowaną w narzędziach MS Office (tam jest ona dużo bardziej zaawansowana).

\texttt{TeXnicCenter} korzysta ze słowników do pobrania ze strony \texttt{openoffice} (\url{https://extensions.openoffice.org/}).
Aby sprawę uprościć słowniki dla języka polskiego (pliki \texttt{pl\_PL.aff} oraz \texttt{pl\_PL.dic}) dołączono do szablonu (są w katalogu \texttt{Dictionaries}). Pliki te należy umieścić w katalogu \texttt{C:\textbackslash{}Program Files\textbackslash{}TeXnicCenter\textbackslash{}Dictionaries}, a w konfiguracji projektu (\texttt{Tools/Options/Spelling}) należy wybrać \texttt{Language: pl}, \texttt{Dialect: PL}. Jeśli główny tekst pracy pisany jest w innym języku, to trzeba zmienić słownik.

Zaskakujące może jest to, że \emph{spell checker} w \texttt{TeXnicCenter} działa zarówno przy pracy na plikach UTF8, jak i na plikach ANSI. Jeśli byłyby jakieś problemy ze słownikiem wynikające z~kodowania znaków, wtedy słownik trzeba przekodować. To powinno pomóc.

\section{Wersjonowanie}
W trakcie edytowania pracy w systemie latex dobrą praktyką jest wersjonowanie tworzonego kodu. Do wersjonowania zaleca się wykorzystać system \texttt{git}. Opis sposobu pracy z tym systemem opisano w licznych tutorialach dostępnych w sieci. Szczególnie godnym polecenia zasobem jest strona domowa projektu \url{https://git-scm.com/}.

Zwykle pracę z systemem wersjonowania rozpoczyna się od utworzenia repozytorium zdalnego i lokalnego. Lokalne służy do bieżącej pracy, zdalne -- do współpracy z innymi użytkownikami (z promotorem). 

Po utworzeniu repozytorium lokalnego i roboczej gałęzi (czy będzie to \texttt{master} czy inna gałąź -- ustalają to sami zainteresowani) należy skopiować do niego wszystkie pliki dostarczone w szablonie, a po ich wstępnym przeredagowaniu należy je zaznaczyć do wersjonowania. Potem należy wysłać zmiany na repozytorium zdalne (możliwa jest też ścieżka odwrotna, można zacząć od zmian na repozytorium zdalnym, które pobrane będą do repozytorium lokalnego).

Podczas kompilowania projektu będą powstawały pliki pomocnicze. Plików tych nie należy wersjonować (zabierają niepotrzebnie miejsce, a przecież zawsze można je odtworzyć uruchamiając kompilację na źródłach). \texttt{git} posiada mechanizm automatycznego odrzucania plików niepodlegających wersjonowaniu. Mechanizm ten bazuje na wykorzystaniu pliku konfiguracyjnego \texttt{.gitignore} zamieszczonego w katalogu głównym repozytorium. O szczegółach \texttt{.gitignore} można poczytać  na stronie \url{https://git-scm.com/docs/gitignore}. 

W sieci można znaleźć liczne propozycje plików konfiguracyjnych \texttt{.gitignore} dopasowanych do potrzeb latexowej kompilacji. Nie trzeba ich jednak szukać. W~szablonie zamieszczono przykładowy taki plik. Zawiera on, między innymi, wpis zabraniający wersjonowania pliku wynikowego \texttt{Dyplom.pdf}. 


Plik \texttt{.gitignore} należy umieścić w repozytorium zaraz po jego utworzeniu. Jeśli w repozytorium pojawiły się już jakieś zmiany zanim pojawił się tam \texttt{.gitignore}, to wtedy należy wykonać kroki opisane na stronie: \url{https://stackoverflow.com/questions/38450276/force-git-to-update-gitignore/38451183}

\begin{lstlisting}[basicstyle=\small\ttfamily]
> > > > You will have to clear the existing git cache first.
    git rm -r --cached .
> > > > Once you clear the existing cache, adds/stages all of the files in the current directory and commit
    git add .
    git commit -m "Suitable Message"
> > > > 
\end{lstlisting}


Zalecany schemat współpracy dyplomanta z promotorem polega na wykonywaniu w kolejnych iteracjach następujących kroków:
\begin{itemize}
\item dyplomant edytuje wybraną część pracy, a po skończeniu edycji wrzuca zmiany do zdalnego repozytorium,
\item dyplomant informuje promotora o zakończeniu etapu prac,
\item dyplomant może zacząć edycję kolejnego fragmentu pracy, a w tym czasie promotor może dokonać oceny/korekty zmian pobranych ze zdalnego repozytorium,
\item promotor wrzuca dokonane przez siebie zmiany do zdalnego repozytorium, informując o tym dyplomanta.
\end{itemize}
Główna zasada tego schematu polega na niedoprowadzaniu do konfliktów (nadpisywania zmian). Jeśli jednak takie konflikty nastąpią, można je niwelować poprzez odpowiednie scalanie zmian (merdżowanie). Swoją drogą, niezłym narzędziem do porównywania zawartości plików i~usuwania niezgodności jest \texttt{WinMerge} (\url{https://winmerge.org/}).


Do wzajemnych powiadomień można wykorzystać pocztę elektroniczną. Można też spróbować wdrożyć mechanizm zatwierdzania zmian (ang.~\emph{merge requests}). Można też umówić się na sprawdzanie zawartości repozytorium zgodnie z jakimś przyjętym harmonogramem. Ważne, by wiadomo było obu stronom, na jakim schemacie współpracy mają bazować.

Dobrą praktyką jest też wstawianie w kod komentarzy. Przyjętą powszechnie konwencją jest rozpoczynanie komentarzy od:
\begin{itemize}
\item \verb|% TO DO: tekst zalecenia| 
-- jeśli jest to jakieś zalecenie promotora, czy też 
\item \verb|% DONE: tekst wyjaśnienia| 
-- jeśli jakieś zalecenie zostało wykonane przez dyplomanta. 
\end{itemize}

Jako zdalne repozytorium można wykorzystać: \texttt{github}, \texttt{bitbucket}, \texttt{gitlab} (są to serwisy, które pozwalają zarządzać repozytoriami \texttt{git}).
Dobrą praktyką jest też uruchomienie klientów git oferujących graficzny interfejs (jak \texttt{SourceTree}, \texttt{GitCracken} itp.). W narzędziach tych można zobaczyć natychmiast na czym polegały wprowadzone w repozytorium zmiany. 